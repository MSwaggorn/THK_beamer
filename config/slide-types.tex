% Custom Slide Type Definitions
% This file provides helper commands for common slide layouts

%=============================================================================
% 1. TITLE SLIDE
%=============================================================================
% Title slide is handled by the standard \titlepage command
% Just use:
% \begin{frame}[plain]
% \titlepage
% \end{frame}

%=============================================================================
% 2. BULLET POINTS SLIDE
%=============================================================================
% Standard frame with itemize environment
% Just use standard beamer frame with \frametitle and itemize
% Example:
% \begin{frame}
% \frametitle{Your Title}
% \begin{itemize}
%     \item First level
%     \begin{itemize}
%         \item Second level (THRed)
%         \begin{itemize}
%             \item Third level (THOrange)
%             \begin{itemize}
%                 \item Fourth level (THPurple)
%             \end{itemize}
%         \end{itemize}
%     \end{itemize}
% \end{itemize}
% \end{frame}

%=============================================================================
% 3. FIGURE SLIDE
%=============================================================================
% Helper environment for single figure slides
\newenvironment{figureframe}[1][0.7]{%
    % #1 = width factor (default 0.7)
    \begin{figure}
    \centering
    \newcommand{\figwidth}{#1}
}{%
    \end{figure}
}

% Usage:
% \begin{frame}
% \frametitle{Figure Title}
% \begin{figureframe}[0.8]
%     \includegraphics[width=\figwidth\textwidth]{image-name}
%     \caption{Your caption text}
% \end{figureframe}
% \end{frame}

%=============================================================================
% 4. SIDE-BY-SIDE SLIDE
%=============================================================================
% Helper commands for two-column layouts
% Use the standard columns environment with equal spacing

% Convenience command for balanced columns
\newcommand{\thkcolumnsep}{0.04\textwidth}
\newcommand{\thkcolumnwidth}{0.48\textwidth}

% Usage:
% \begin{frame}
% \frametitle{Side-by-Side Title}
% \begin{columns}[T]
%     \begin{column}{\thkcolumnwidth}
%         % Left content (itemize, figure, etc.)
%         \begin{itemize}
%             \item Point 1
%             \item Point 2
%         \end{itemize}
%     \end{column}
%     \hfill
%     \begin{column}{\thkcolumnwidth}
%         % Right content (itemize, figure, etc.)
%         \begin{figure}
%             \centering
%             \includegraphics[width=\textwidth]{image}
%             \caption{Caption}
%         \end{figure}
%     \end{column}
% \end{columns}
% \end{frame}

%=============================================================================
% 5. SIDE-BY-SIDE WITH SUBHEADINGS
%=============================================================================
% Use the \subheading command defined in fonts.tex for column headers

% Usage:
% \begin{frame}
% \frametitle{Main Title}
% \begin{columns}[T]
%     \begin{column}{\thkcolumnwidth}
%         \subheading{Left Subheading}
%         \begin{itemize}
%             \item Point 1
%             \item Point 2
%         \end{itemize}
%     \end{column}
%     \hfill
%     \begin{column}{\thkcolumnwidth}
%         \subheading{Right Subheading}
%         \begin{itemize}
%             \item Point 1
%             \item Point 2
%         \end{itemize}
%     \end{column}
% \end{columns}
% \end{frame}

%=============================================================================
% ADDITIONAL HELPER COMMANDS
%=============================================================================

% Command for tables with proper styling
\newcommand{\tableframe}[1]{%
    \begin{table}
    \centering
    #1
    \end{table}
}

% Example table usage with booktabs:
% \begin{frame}
% \frametitle{Table Example}
% \tableframe{
%     \begin{tabular}{lcc}
%     \toprule
%     Item & Value 1 & Value 2 \\
%     \midrule
%     A & 1.23 & 4.56 \\
%     B & 7.89 & 0.12 \\
%     \bottomrule
%     \end{tabular}
%     \caption{Table caption}
% }
% \end{frame}
